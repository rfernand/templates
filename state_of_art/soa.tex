\documentclass[letter, 10pt]{IEEEtran}
\usepackage{graphicx}
\usepackage{url}
\usepackage{hyperref}
\usepackage[utf8]{inputenc}
\usepackage{enumerate}
\usepackage{amsfonts}
\usepackage{amsmath}
\usepackage{color}

\definecolor{red}{rgb}{1,0,0}
\definecolor{green}{rgb}{0,1,0}
\definecolor{blue}{rgb}{0,0,1}
\newcommand{\blue}{\textcolor{blue}}
\newcommand{\red}{\textcolor{red}}
\newcommand{\green}{\textcolor{green}}

\title{Título}

\author{Rodrigo G. Fernández Gaete\\\url{rfernand@csrg.cl}\\
Computer Systems Research Group~\cite{CSRG}\\ Universidad Técnica Federico Santa
María, Valparaíso, Chile}

%\author{
%\IEEEauthorblockN{
%Rodrigo G. Fernández\IEEEauthorrefmark{1}
%}
%\IEEEauthorblockA{
%\IEEEauthorrefmark{1}Computer Systems Research Group~\cite{CSRG}, Universidad Técnica Federico Santa María, Av. ~España~1680, Valparaíso, Chile
%\IEEEauthorrefmark{2}Centro de Robótica~\cite{CR}, Universidad Técnica Federico Santa María, Av. ~España~1680, Valparaíso, Chile
%}
%}
\begin{document}
\bibliographystyle{plain}
\pagestyle{empty}

\maketitle\thispagestyle{empty}

\begin{abstract}
Este documento presenta
%el avance y proceso de creación de una aplicación que permita desarrollar las
%rutinas de control para diferentes tipos de plataformas robóticas, unificando los diferentes mecanismos de comunicación
%con cada dispositivo en un solo lenguaje de programación, buscando simplificar la programación de las
%plataformas robóticas y hacer más eficiente el trabajo del investigador.
%Dentro de las consideraciones se encuentran: (a) la elección de un lenguaje estandarizado que posibilite un fácil
%aprendizaje y permita realizar una amplia gama de tareas; (b) la estructura del software a desarrollar, y (c)
%la perduración del mismo en el tiempo.
%Tras haber considerado todos estos puntos, se estableció que nuestro software sería desarrollado en
%Python, con una estructura modular y extensible utilizando archivos XML, y de código abierto.
%Se presentan resultados parciales sobre el desarrollo del proyecto y del software.
\end{abstract}

\textbf{Palabras Clave:}  key1, key2, key3

\section{Introducción}
\label{sec:intro}
% Esta mala, falta revisarla una vez terminar el paper.

El siguiente documento tiene como objetivo principal, 
%presentar el estudio realizado en distintas plataformas robóticas,
%particularmente Arduinos~\cite{ARDUINO},
%teniendo como finalidad sustentar en forma sólida,
%el desarrollo de la solución a la siguiente problemática: \\

\emph{
Actualmente,
 (...) ,
se tiene como finalidad:
\begin{enumerate}[a.]
    \item 
\end{enumerate}
}

Por tal motivo, en el presente documento se detallarán los siguientes puntos:

\begin{itemize}
	\item Señalar la problemática que actualmente afecta al 
    %Centro de Robótica UTFSM, centro de investigación con el cual se está trabajando en forma colaborativa, para poder establecer un Software que supla las necesidades reales.
	\item Detalle de todas las posibles alternativas que darían solución a 
    %la problemática planteada anteriormente, las cuales fueron analizadas con el fin de desarrollar la más óptima.
	\item Descripción de distintos tipos de técnicas de
    %desarrollo de software, utilizado para construir la solución.
	\item Método de implementación de 
    %la solución a desarrollar.
\end{itemize}

El presente trabajo se divide como sigue:
La sección \ref{sec:conclusiones} presenta un resumen
%y consideraciones de nuestro trabajo a la fecha,
%además de enfocarse a delinear el futuro de nuestro proyecto.


\section{Estado del Arte}
\label{sec:estado}
El trabajo de los investigadores del área de 
%la robótica fuera del diseño del robot en sí,
%es lidiar con el problema de encontrar y/o desarrollar un medio a través del cual interactuar con él.
%Una de las posibles soluciones a las que pueden recurrir, es la reutilización de software,
%sin embargo, esta no es una tarea sencilla~\cite{wcam-1}, ya que es necesario entender el código
%y además adaptarlo a nuestro sistema, tarea que requiere tiempo, el cual podría ser mejor aprovechado
%en otros aspectos de la misma investigación.

Otra de las soluciones actuales es 
%buscar un software que unifique la
%comunicación y la interacción con el robot~\cite{wcam-4}.

Actualmente existen distintas soluciones de software para el estudio de 
%robots, soluciones que van desde
%reproducir sus movimientos a través de la realización de simulaciones de las diferentes plataformas, hasta 
%la implementación de interfaces para la programación y comunicación.

Un punto importante a la hora de realizar comparaciones entre diferentes
aplicaciones es 
%que los enfoques de solución que utilizan son distintos y como
%es señalado en~\cite{wcam-4} se termina intentando realizar una comparación
%entre cosas esencialmente muy diferentes.

Por otro lado, en algunas soluciones,
%como Pyro~\cite{PYRO}, se puede apreciar
%una escasa aplicación de ingeniería de software, debido a la poca preocupación
%por la interfaz de manejo sin aprovechar las facilidades que las tecnologías
%actuales ofrecen, dificultando el trabajo de fondo innecesariamente.

Hay un punto a destacar en el ámbito de 
%los simuladores, en el cual si bien se trata de generar un
%ambiente en el que se desenvuelve una plataforma robótica, de la manera más real posible, nunca se podrá
%determinar todas las posibles variables que pudiesen afectar algún tipo de experimento, por lo que
%idealmente se puede generar un entorno reduciendo variables, logrando así limitar el problema, pero
%dejando de lado toda realidad.

Referente a la propiedad intelectual del software relacionado a la investigación del área de 
%la robótica,
%se está enfrentando a dos grandes tendencias, por un lado se tiene al software OpenSource~\cite{OPENSOURCE}, 
%el cual hace referencia al software que posee características de ser distribuido,
%desarrollado, obtenido e incluido libremente, entre otras características; por el otro lado se encuentra
%el software propietario, el cual limita las posibilidades de usarlo, modificarlo o
%redistribuirlo con o sin modificaciones, como lo es Microsoft Robotics Developer Studio~\cite{MICROSOFT}.
%Dada las características de cada tendencia, se ve que OpenSource permite establecer, dada sus
%características, un total apoyo a la libre investigación y aporte a los diferentes softwares.


\section{Conclusiones}
\label{sec:conclusiones}
\begin{frame}
\frametitle{Conclusiones}
\begin{itemize}
    \item 
\end{itemize}
\end{frame}




\bibliography{article,paper,url}
%\vfill \hfill CMF/GZN/JPI/EBG/RFG/DI/UTFSM/2010
\end{document}
