El trabajo de los investigadores del área de 
%la robótica fuera del diseño del robot en sí,
%es lidiar con el problema de encontrar y/o desarrollar un medio a través del cual interactuar con él.
%Una de las posibles soluciones a las que pueden recurrir, es la reutilización de software,
%sin embargo, esta no es una tarea sencilla~\cite{wcam-1}, ya que es necesario entender el código
%y además adaptarlo a nuestro sistema, tarea que requiere tiempo, el cual podría ser mejor aprovechado
%en otros aspectos de la misma investigación.

Otra de las soluciones actuales es 
%buscar un software que unifique la
%comunicación y la interacción con el robot~\cite{wcam-4}.

Actualmente existen distintas soluciones de software para el estudio de 
%robots, soluciones que van desde
%reproducir sus movimientos a través de la realización de simulaciones de las diferentes plataformas, hasta 
%la implementación de interfaces para la programación y comunicación.

Un punto importante a la hora de realizar comparaciones entre diferentes
aplicaciones es 
%que los enfoques de solución que utilizan son distintos y como
%es señalado en~\cite{wcam-4} se termina intentando realizar una comparación
%entre cosas esencialmente muy diferentes.

Por otro lado, en algunas soluciones,
%como Pyro~\cite{PYRO}, se puede apreciar
%una escasa aplicación de ingeniería de software, debido a la poca preocupación
%por la interfaz de manejo sin aprovechar las facilidades que las tecnologías
%actuales ofrecen, dificultando el trabajo de fondo innecesariamente.

Hay un punto a destacar en el ámbito de 
%los simuladores, en el cual si bien se trata de generar un
%ambiente en el que se desenvuelve una plataforma robótica, de la manera más real posible, nunca se podrá
%determinar todas las posibles variables que pudiesen afectar algún tipo de experimento, por lo que
%idealmente se puede generar un entorno reduciendo variables, logrando así limitar el problema, pero
%dejando de lado toda realidad.

Referente a la propiedad intelectual del software relacionado a la investigación del área de 
%la robótica,
%se está enfrentando a dos grandes tendencias, por un lado se tiene al software OpenSource~\cite{OPENSOURCE}, 
%el cual hace referencia al software que posee características de ser distribuido,
%desarrollado, obtenido e incluido libremente, entre otras características; por el otro lado se encuentra
%el software propietario, el cual limita las posibilidades de usarlo, modificarlo o
%redistribuirlo con o sin modificaciones, como lo es Microsoft Robotics Developer Studio~\cite{MICROSOFT}.
%Dada las características de cada tendencia, se ve que OpenSource permite establecer, dada sus
%características, un total apoyo a la libre investigación y aporte a los diferentes softwares.
