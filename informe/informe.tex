\documentclass[letter, 10pt]{article}
\usepackage[utf8]{inputenc}
\usepackage[spanish]{babel}
\usepackage{amsfonts}
\usepackage{amsmath}
\usepackage[dvips]{graphicx}
\usepackage{url}
\usepackage[top=3cm,bottom=3cm,left=3.5cm,right=3.5cm,footskip=1.5cm,headheight=1.5cm,headsep=.5cm,textheight=3cm]{geometry}
\usepackage{listings}
\usepackage{color}
\usepackage{fancyvrb}
\usepackage{fancyhdr}
\usepackage{booktabs}


\definecolor{red}{rgb}{1,0,0}
\definecolor{green}{rgb}{0,1,0}
\definecolor{blue}{rgb}{0,0,1}
\newcommand{\blue}{\textcolor{blue}}
\newcommand{\red}{\textcolor{red}}
\newcommand{\green}{\textcolor{green}}

%%%%%%%%%%%%%%%%%%%%%%
%Estilo del documento%
%%%%%%%%%%%%%%%%%%%%%%
\pagestyle{fancyplain}

%%%%%%%%%%%%%%%%%%%%%%%%%%%%%%%%%%%%%%%%%%%
%Fancyheadings. Top y Bottom del documento%
%%%%%%%%%%%%%%%%%%%%%%%%%%%%%%%%%%%%%%%%%%%
% Recuerde que en este documento la portada del documento no posee
% numeracion, pero de igual manera llamaremos a esa primera pagina la numero
% 1, y la que viene la dos. Esto es para tener una idea de las que
% llamaremos pares e impares
\lhead{Seminario de Sistemas Distribuidos} %Parte superior izquierda
\rhead{\bf \it Sistemas Electrónicos de Votación} %Parte superior derecha
\lfoot{} %Parte inferior izquierda.
\cfoot{} %Parte inferior central
\rfoot{\bf \thepage} %Parte inferior derecha
\renewcommand{\footrulewidth}{0.4pt} %Linea de separacion inferior

\bibliographystyle{plain}

\begin{document}

%\title{Sistemas de Votación Electrónicos \\ \begin{Large} Implementaciones de
%Código Abierto \end{Large}}
%\author{Rodrigo Fernández \\ 2673002-3 \\ \texttt{rfernand@inf.utfsm.cl}}
%...}
%\date{\today}
%\maketitle
%%%%%%%%%%%%%%%%%%%%%%%%%%
%Definicion de la portada%
%%%%%%%%%%%%%%%%%%%%%%%%%%
\begin{titlepage}
    \begin{center}
	\begin{tabular}{ccc}
	    \includegraphics[width=3cm]{img/utfsm}
	    & 
	    \hspace{-0.2cm}
	    \begin{tabular}{c}
		Universidad Técnica Federico Santa María \\ \hline
		\vspace{0.2cm}
		Departamento de Informática\\
		\vspace{1.2cm}
	    \end{tabular}
	    \hspace{0.2cm}
	    &
            \includegraphics[width=2cm]{img/di}
	\end{tabular}

	\vspace{1cm}
	%Titulo del Documento
	\begin{tabular}{c}
		\Huge{\sc{Sistemas Electrónicos de Votación}}\\
\\
        
		\huge{\sc{Implementaciones de Código Abierto}}\\
        \\

		\includegraphics[scale=0.4]{img/voting} \\\\
	\vspace{1cm}
	\end{tabular}

	\begin{tabular}{c}
		\large{\sc{Integrantes}}
	\end{tabular}
	\\
	\begin{tabular}{cr}
         	\normalsize{Rodrigo Fern\'andez Gaete}  & \footnotesize \emph{Tel: +56 9 8419 3413}\\
						        & \footnotesize \url{rfernand@inf.utfsm.cl}\\
	\end{tabular}\\
	\vspace{1cm}
	\input{resumen}
	\vspace{0.6cm}
	%Fecha
		\normalsize{\sc{\today}}\\
    \end{center}
\end{titlepage}


\section{Descripción General del Tema}
\input{src/1-antecedentes}

\section{Objetivos}
\input{src/2-objetivos}


\section{Plan de Trabajo}
\input{src/3-gantt}

\section{Recursos}
\input{src/4-recursos}

\section{Trabajo Adelantado}
\input{src/5-adelanto}

\newpage
\bibliography{src/paper,src/url}


\end{document}
