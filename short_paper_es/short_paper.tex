% Porfavor no cambiar el formato, este es el exigido por el Encuentro Chileno de la Computacion.
% Mantenerlo así hasta el 15/07/2008.
% Idioma: español, inglés o portugués.
% Tamaño 8 1/2" x 11" (carta, letter USA), máximo 10 páginas, incluyendo resumen no superior a
% 130 palabras.
% Márgenes: superior 3,5 cm, inferior 2 cm, derecho e izquierdo: 2,5 cm.
% Formato: Texto justificado a la derecha e izquierda, páginas no numeradas.
% Tipo de letra: Times 10 para el texto y 12 para títulos y apartados.
% La primera página debe contener: El título del trabajo, nombre completo de autores, afiliación y
% direcciones, así como el resumen del trabajo y palabras claves para su clasificación sobre la base de
% los temas propuestos en este llamado.

\documentclass[conference]{IEEEtran}
%\documentclass[letter, 10pt]{IEEEtran}
\usepackage{graphicx}
\usepackage{url}
\usepackage[utf8]{inputenc}
\usepackage{enumerate}
\usepackage{amsfonts}
\usepackage{amsmath}
\usepackage{color}

\definecolor{red}{rgb}{1,0,0}
\definecolor{green}{rgb}{0,1,0}
\definecolor{blue}{rgb}{0,0,1}
\newcommand{\blue}{\textcolor{blue}}
\newcommand{\red}{\textcolor{red}}
\newcommand{\green}{\textcolor{green}}

%opening
\title{Titulo}

\author{
\IEEEauthorblockN{
Rodrigo G. Fernández\IEEEauthorrefmark{1}
}
\IEEEauthorblockA{
\IEEEauthorrefmark{1}Computer Systems Research Group~\cite{CSRG}, Universidad Técnica Federico Santa María, Av. ~España~1680, Valparaíso, Chile
%\IEEEauthorrefmark{2}Centro de Robótica~\cite{CR}, Universidad Técnica Federico Santa María, Av. ~España~1680, Valparaíso, Chile
}
}
\begin{document}
\bibliographystyle{plain}
\pagestyle{empty}

\maketitle\thispagestyle{empty}

\begin{abstract}
Este documento presenta
%el avance y proceso de creación de una aplicación que permita desarrollar las
%rutinas de control para diferentes tipos de plataformas robóticas, unificando los diferentes mecanismos de comunicación
%con cada dispositivo en un solo lenguaje de programación, buscando simplificar la programación de las
%plataformas robóticas y hacer más eficiente el trabajo del investigador.
%Dentro de las consideraciones se encuentran: (a) la elección de un lenguaje estandarizado que posibilite un fácil
%aprendizaje y permita realizar una amplia gama de tareas; (b) la estructura del software a desarrollar, y (c)
%la perduración del mismo en el tiempo.
%Tras haber considerado todos estos puntos, se estableció que nuestro software sería desarrollado en
%Python, con una estructura modular y extensible utilizando archivos XML, y de código abierto.
%Se presentan resultados parciales sobre el desarrollo del proyecto y del software.
\end{abstract}

\textbf{Palabras Clave:} Testing, Desarrollo de Software

\section{Introducción}
\label{sec:intro}
% Esta mala, falta revisarla una vez terminar el paper.

El siguiente documento tiene como objetivo principal, 
%presentar el estudio realizado en distintas plataformas robóticas,
%particularmente Arduinos~\cite{ARDUINO},
%teniendo como finalidad sustentar en forma sólida,
%el desarrollo de la solución a la siguiente problemática: \\

\emph{
Actualmente,
 (...) ,
se tiene como finalidad:
\begin{enumerate}[a.]
    \item 
\end{enumerate}
}

Por tal motivo, en el presente documento se detallarán los siguientes puntos:

\begin{itemize}
	\item Señalar la problemática que actualmente afecta al 
    %Centro de Robótica UTFSM, centro de investigación con el cual se está trabajando en forma colaborativa, para poder establecer un Software que supla las necesidades reales.
	\item Detalle de todas las posibles alternativas que darían solución a 
    %la problemática planteada anteriormente, las cuales fueron analizadas con el fin de desarrollar la más óptima.
	\item Descripción de distintos tipos de técnicas de
    %desarrollo de software, utilizado para construir la solución.
	\item Método de implementación de 
    %la solución a desarrollar.
\end{itemize}

El presente trabajo se divide como sigue:
La sección \ref{sec:conclusiones} presenta un resumen
%y consideraciones de nuestro trabajo a la fecha,
%además de enfocarse a delinear el futuro de nuestro proyecto.


\section{Formalización del Problema}
\label{sec:formalizacion}
\input{src/2-formalizacion}

\section{Soluciones Propuestas}
\label{sec:solucion}
\input{src/3-solucion}

\section{Pruebas}
\label{sec:pruebas}
\input{src/4-pruebas}

\section{Resultados}
\label{sec:resultados}
\input{src/5-resultados}

\section{Trabajo Relacionado}
\label{sec:soa}
\input{src/6-soa}

\section{Conclusiones}
\label{sec:conclusiones}
\begin{frame}
\frametitle{Conclusiones}
\begin{itemize}
    \item 
\end{itemize}
\end{frame}



%\section*{Agradecimientos}
%
%  Este trabajo es financiado por
% (...)
%  Por otro lado, los autores agradecen el apoyo de CSRG  (Computer Systems Research Group)
%del Departamento de Informática de la UTFSM, por facilitarnos su infraestructura y respaldo en el área técnica.
%  De la misma forma, los autores agradecen también al Centro de Robótica de la UTFSM,
%por el interés en la participación colaborativa para poder realizar nuestra investigación y
%desarrollo del proyecto.
%  Finalmente, los autores agradecen la revisión del comité de evaluación por sus útiles comentarios acerca de éste documento.



\bibliography{article,paper,url}
\vfill \hfill CMF/GZN/JPI/EBG/RFG/DI/UTFSM/2010
\end{document}

%%% Local Variables: 
%%% mode: latex
%%% TeX-master: t
%%% End: 
