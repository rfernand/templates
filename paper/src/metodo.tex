Hoy en día existen muchas aplicaciones y recursos robustos para poder solucionar
distintos problemas enfocados al estudio y análisis de 
%plataformas robóticas.

Para desarrollar este proyecto fue necesario detectar cuáles de estos recursos
son los que permitirían
%crear una Interfaz Gráfica de Usuario (GUI) que pueda ofrecerle
%a los investigadores una alternativa que facilite su labor.

Por tanto, fue una necesidad el realizar un estudio de 
%las herramientas que otras interfaces
%ofrecían \footnote{\emph{Descritas posteriormente en VI. Resultados, subsección A}}, cumpliendo así una
%labor similar, tomando en cuenta que nuestro proyecto define como
%usuario final a desarrolladores e investigadores interiorizados en el área de la robótica.

Para poder obtener dicha información fue necesario 
%dirigirnos al origen de la problemática y establecer
%un trabajo en conjunto con las personas dedicadas al área de la robótica. En nuestro caso, la
%investigación se ha realizado en conjunto con el Centro de Robótica de la UTFSM. Ésto 
%ayudó a definir correctamente el enfoque que se tendría que dar al proyecto, tanto por el lado de
%los requerimientos principales de nuestro proyecto, como de la interfaz que un investigador de esta área
%espera encontrar para, finalmente, facilitar la realización de la investigación en las plataformas
%robóticas.

%PROPÓSITO: CONVENCER A UNA AUDIENCIA TÉCNICA QUE SE SABE CÓMO CONSTRUIR LA SOLUCIÓN.
Es en base a este enfoque que, después de realizar una investigación sobre las diferentes herramientas,
se decidió
%para el proyecto utilizar el lenguaje de programación Python, por el entorno de trabajo
%que nos ofrece, permitiéndonos generar soluciones rápidamente y ser un lenguaje multiparadigma fácilmente
%extensible.
%Además, posee bastantes facilidades para trabajar en la implementación de interfaces de éste tipo, ya que
%contiene una gran cantidad de bibliotecas para el manejo de diferentes aspectos del proyecto, como la
%comunicación serial (pySerial~\cite{PYSERIAL}),
%creación de algoritmos, y comunicación con la interfaz gráfica (pyQt~\cite{PYQT}), la cual junto a bibliotecas
%específicas, ofrece herramientas con un entorno para desarrollar Interfaces
%Gráficas de Usuario (Qt Designer~\cite{QTDESIGNER}).

Para 
%las bibliotecas gráficas, se optó por utilizar QT, al ser de fácil implementación en conjunto con Python (pyQt) y por su gran portabilidad
%entre diferentes sistemas operativos.
Finalmente,
%para todo lo que significa la configuración de las plataformas
%robóticas, se utilizó XML, por ser un estándar para el intercambio de
%información estructurada entre diferentes plataformas, además de poseer la ventaja
%de ser extensible para poder modificar y usar sus datos sin mayores
%complicaciones. De esta manera, se reduce la dificultad de aprendizaje y
%facilita el trabajo para que personas externas puedan aportar al
%desarrollo del proyecto.

Para desarrollar el presente trabajo, se han escogido métodos 
%ágiles, sobre todo en extreme programming~\cite{XP},
%por los principios que la sustentan; \emph{simplicidad, comunicación, retroalimentación y coraje}\cite{xp-book}.

Para complementar y corregir las deficiencias del presente proyecto,
%se utiliza como sistema de seguimiento de desarrollo de software a
%Trac~\cite{TRAC}, el cual integra una wiki, un repositorio GIT~\cite{GIT} y
%herramientas para el seguimiento y control de tareas. Esto permitió un rápido
%avance en las tareas del grupo, pues cada integrante podía estar al tanto de
%las actividades de sus compañeros de equipo, sin necesidad de mantener
%reuniones presenciales. Cabe mencionar que ésta metodología de trabajo
%nos fue de fácil implementación, gracias a que cada uno de los miembros del
%equipo se encontraban previamente familiarizados con las herramientas utilizadas, a que
%las tareas a realizar eran fácilmente identificables y que comenzadas las
%etapas de desarrollo no requerían de un gran trabajo en conjunto.
