% Esta mala, falta revisarla una vez terminar el paper.

El siguiente documento tiene como objetivo principal, 
%presentar el estudio realizado en distintas plataformas robóticas,
%particularmente Arduinos~\cite{ARDUINO},
%teniendo como finalidad sustentar en forma sólida,
%el desarrollo de la solución a la siguiente problemática: \\

\emph{
Actualmente,
 (...) ,
se tiene como finalidad:
\begin{enumerate}[a.]
    \item 
\end{enumerate}
}

Por tal motivo, en el presente documento se detallarán los siguientes puntos:

\begin{itemize}
	\item Señalar la problemática que actualmente afecta al 
    %Centro de Robótica UTFSM, centro de investigación con el cual se está trabajando en forma colaborativa, para poder establecer un Software que supla las necesidades reales.
	\item Detalle de todas las posibles alternativas que darían solución a 
    %la problemática planteada anteriormente, las cuales fueron analizadas con el fin de desarrollar la más óptima.
	\item Descripción de distintos tipos de técnicas de
    %desarrollo de software, utilizado para construir la solución.
	\item Método de implementación de 
    %la solución a desarrollar.
\end{itemize}

El presente trabajo se divide como sigue:
En la sección~\ref{sec:definicion}
%acota la problemática de la programación de plataformas robóticas, 
%haciendo notar las dificultades que se tienen al llevar a cabo esta tarea.
La sección~\ref{sec:estado} muestra un resumen de los trabajos
%que se han realizado en torno a este problema,
%y las distintas aproximaciones que se han adoptado en el último tiempo.
En la sección~\ref{sec:objetivos}
%se muestran los delineamientos del trabajo que se realizará
%dentro de este proyecto.
La sección~\ref{sec:metodo} muestra cómo se lleva a cabo el proyecto,
%cuales han sido las principales decisiones,
%y cómo se avanzará.
En la sección~\ref{sec:resultados}
%se muestran los resultados obtenidos hasta el momento
%en cuanto al desarrollo de la interfaz
%y los puntos trabajados en conjunto con el Centro de Robótica UTFSM.
La sección \ref{sec:conclusiones} presenta un resumen
%y consideraciones de nuestro trabajo a la fecha,
%además de enfocarse a delinear el futuro de nuestro proyecto.
